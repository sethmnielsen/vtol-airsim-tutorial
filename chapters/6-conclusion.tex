%TC:envir minttcb [ignore] xall

\chapter{Conclusions and Future Work}\label{chp:conclusion}

Autonomous eVTOL aircraft have the potential to provide a new mode of air transportation for humans and cargo that is cheaper, more rapid and more efficient than has previously been possible. Simulation is key to the development of these aircraft, but there is currently a scarcity of simulation tools that have high-quality graphics, are publicly accessible, are easy to use, and which can simulate eVTOLs.

In this work, we contribute \textit{VTOL-AirSim}, a full integration of eVTOL aircraft into Microsoft AirSim. Our work includes new visual components that are packaged with VTOL-AirSim: a fully animated tiltrotor mesh, and a realistic city environment. We also contribute a tutorial on how to use VTOL-AirSim, and a guide on how to extend the capabilities of VTOL-AirSim. Finally, we demonstrate complete examples of creating custom aircraft and environments for VTOL-AirSim, so that others may customize it according to their needs.

This work aims to make VTOL-AirSim as accessible as possible to those seeking to use the simulator and to those who wish to increase its capabilities. It is our hope that others will find VTOL-AirSim to be a useful tool for simulating eVTOL aircraft in visually realistic environments. VTOL-AirSim has already been used in several research projects at the BYU MAGICC Lab. We hope that this trend will continue, and that future work will increase its reach even further.

\section{Review of Contributions}

A user guide for VTOL-AirSim can be found in Chapter~\ref{chp:userguide}. We give an overview of the additions made by VTOL-AirSim in Section~\ref{sec:vtolairsim} and its unique settings in Section~\ref{sec:settings_file}. In Section~\ref{sec:vtol_examples}, we give three complete examples of how to fly eVTOL aircraft: how to interface with a trajectory generator, geometric controller, and control allocation module in VTOLsim to send motor PWM commands (Section~\ref{sec:vtol_geometric_control}); how to set the vehicle's state via teleporting with an external dynamics simulation (Section~\ref{sec:vtol_teleport}); and how to interface with the PX4 Autopilot's VTOL controller to fly a mission (Section~\ref{sec:vtol_px4}).

In Chapter~\ref{chp:extending}, we show how to set up and use all the tools required for developing and extending VTOL-AirSim. Section~\ref{sec:unreal_setup} explains how to set up Unreal Engine on Linux, and Section~\ref{sec:vtolairsim_setup} explains how to set up VTOL-AirSim for development work. In Section~\ref{sec:vtolairsim_unreal}, we give a basic guide on working with VTOL-AirSim in the Unreal Editor.

We give complete examples of how to customize VTOL-AirSim with custom aircraft and custom environments in Chapters~\ref{chp:custom_aircraft} and~\ref{chp:custom_envs}, respectively. In Chapter~\ref{chp:custom_aircraft}, we show how to: obtain an aircraft mesh (Section~\ref{sec:meshes}), edit a mesh in Blender (Section~\ref{sec:blender_example}), import the mesh into Unreal Engine (Section~\ref{sec:import_mesh_ue4}), make a new Blueprint (Section~\ref{sec:blueprint}), and finally create a new \CC class for the custom aircraft (Section~\ref{sec:cpp_class}). Chapter~\ref{chp:custom_envs} details how to use the Epic Games Launcher on Windows to obtain assets from the Unreal Engine Marketplace (Section~\ref{sec:obtain_assets_windows}) and how to edit an Unreal Project with a custom environment in Linux using the Unreal Editor (Section~\ref{sec:edit_project_ueditor}).


\section{Future Work}
Our work of creating VTOL-AirSim is only the beginning of what can be done for the simulation of eVTOL aircraft in AirSim. There are a few ways in which VTOL-AirSim can be improved, and there are many things that can be added that would greatly increase its utility for several types of research projects. We present here a few examples of improvements that can be made.

First, the teleport functionality in VTOL-AirSim only works for setting the pose of the vehicle. Setting the tilt of the rotors with the teleport command, \ci{simSetVtolPose}, is broken. We attempted to implement this in code via a new function, named \ci{setPoseCustom}, to the \ci{TiltrotorPawnSimApi} \CC class. This function does successfully set the tiltrotor's pose in the data structure holding its physics state, but it appears that the current implementation for setting the rotor tilts is only successful for a brief moment until the motor outputs are overwritten in \ci{vtol_simple::Firmware::update()}, specifically by the call to \ci{board_->writeOutput()}. The firmware is updated at every iteration of AirSim's main loop, and since the firmware hasn't received any commands, it repeatedly sets the rotor tilts to zero, i.e., their nominal angle. Commands made by the AirSim client are executed on a separate thread from the main loop, and thus the two threads compete with one another to produce instantaneous, rapid shifting of the rotor tilts between their set values and their nominal angle. A solution could be to bypass the \ci{VtolSimple} firmware entirely, or to find a way to work with the firmware to set new motor outputs.

Second, while VTOL-AirSim includes support for tiltrotor aircraft, this is just one of several types of winged eVTOL. It also contains only one specific eVTOL model, the E-flite Convergence. Adding a new aircraft model is fairly straightforward, though it requires modifying the VTOL-AirSim source code and performing a source build. This is also how AirSim implements different multirotor models, so this is not unique to VTOL-AirSim; but it is more cumbersome than it needs to be. Ideally, the Convergence model would be the default, and a user could specify a parameter file that would be read at runtime to change which internal aircraft model is used. Adding support for other types of eVTOL would be a bit more difficult, but we designed VTOL-AirSim with adding other eVTOL types in mind, and there are empty functions left in the code that simply need to be implemented for another eVTOL type and it should work. The more time-intensive component would be to add a new aircraft mesh to represent the new eVTOL type, and if it is to be built in to VTOL-AirSim, how to specify which eVTOL type is desired in the settings file and then dynamically change the mesh of the vehicle.

Finally, it would be ideal for VTOL-AirSim to use a tiltrotor mesh which represents the actual dynamic model. This could be a different mesh which is a true tri-tiltrotor like the E-flite Convergence, or this could be a new dynamic model for a dual-tiltrotor, or a combination of both solutions.
