%TC:envir minttcb [ignore] xall

\chapter{Miscellaneous Instructions}\label{apdx:misc_instructions}

% \section{Add a New Aircraft Configuration}\label{apdx:add_configuration}
% This section explains how one could change or add a new \textit{configuration} for the \verb|VTOL| vehicle type. A configuration is the

\section{Authentication for GitHub Using \texttt{gh} Client}\label{apdx:github_config}

An easy way to configure your Linux environment to clone, push, pull, etc.\ from GitHub repositories is to use the official GitHub CLI tool called \ci{gh}. First, go to \url{https://github.com/cli/cli/blob/trunk/docs/install_linux.md} and follow the instructions to install \ci{gh}. Then, in a terminal, run the command \ci{gh auth login}, which will guide you through a number of steps, for which you should do the following:

\begin{itemize}
    \item Choose \ci{GitHub.com}, then choose \ci{SSH}
    \begin{itemize}
        \item If you haven't generated a public SSH key previously, then choose ``yes'' at the prompt by pressing Enter, then press Enter again to leave passphrase blank
        \item If you already have a public SSH key, then choose it when prompted
    \end{itemize}
    \item Choose \ci{Login with a web browser}
    \item Copy the given code, then press Enter; this will open your web browser
    \item Authorize in web browser using the copied code
    \item Go back to the terminal, then press Enter to finish
\end{itemize}

You can now use either \ci{git} commands or \ci{gh} commands to interact with GitHub repositories. For example, to clone the AirSim repository, you can run either of the following:

\begin{minttcb}[title={Clone a GitHub Repository Option 1: \texttt{git}}]{bash}
    git clone git@github.com:microsoft/AirSim
\end{minttcb}
\begin{minttcb}[title={Clone a GitHub Repository Option 2: \texttt{gh}}]{bash}
    gh repo clone microsoft/AirSim
\end{minttcb}

For more information about connecting to GitHub with SSH, see the GitHub docs at \url{https://docs.github.com/en/authentication/connecting-to-github-with-ssh}.

\section{Advanced Setup for Python Virtual Environments}\label{apdx:adv_setup_python_venv}

You can place these bash functions in either your \ci{.bashrc} file or \ci{.zshrc} file, or in another file which is sourced by either of those files. Then, you can run, for example, the command \ci{workon airsim} to activate your Python virtual environment named \ci{airsim}. Run the command \ci{workoff} to deactivate your virtual environment.

\begin{minttcb}[]{bash}
    workon() {
        if [[ -n "$VIRTUAL_ENV" ]]; then
            deactivate
        fi
        source $HOME/.virtualenvs/$1/bin/activate
    }

    workoff() {
        if [[ -n "$VIRTUAL_ENV" ]]; then
            deactivate
        fi
    }
\end{minttcb}


\section{Alternate Method for Moving Mesh Origins in Blender}\label{apdx:blender_origins_alt}
In the past, we have accomplished getting the correct pivot points in Unreal Engine using a different method than the procedure outlined in Section~\ref{sec:blender_example}. The method is to:

\begin{itemize}
    \item create separate \ci{.blend} Blender files for each object mesh that you want to animate on the aircraft (e.g., engines, propellers, ailerons)
    \item Do the following for each Blender file:
    \begin{itemize}
        \item If there are multiple meshes, choose one as the root object and make the other objects the children of it using the drag, then \texttt{Shift+Alt}, then drop technique
        \item Hold pointer over the Viewport and press \texttt{A} to select all objects
        \item Select \textbf{Object > Set Origin > Origin to Geometry} to set origins of the meshes to their geometric centers
        \item Select \textbf{Object > Snap > Cursor to World Origin}
        \item Select the object, or, if there are multiple objects, select the root object
        \item Select \textbf{Object > Snap > Selection to Cursor}
        \item If there are any nonzero values for the Location of the object (or root object), apply the Location by holding the pointer over the Viewport and pressing \texttt{Ctrl+A} then clicking \textbf{Location}
        \item Repeat the process to apply all Rotations
        \item Export to FBX file with default settings under the \textbf{Transform} section:
        \begin{itemize}
            \item \textbf{Up} set to \textbf{Y Up}
            \item \textbf{Use Space Transform} checked
            \item \textbf{Apply Transform} unchecked
        \end{itemize}
    \end{itemize}
    \item After finishing the exports, import each FBX file into UE4
    \item Create a Blueprint with the aircraft's body as the Root Component
    \item Add the other meshes to the Blueprint
    \item Manually set each component's Location to where they should be relative to the aircraft body
\end{itemize}
