% This is to make sure that if it goes onto multiple pages that it will only be on odd pages.
\afterpage{\cleardoublepage}
% Some people have had problems with needing a little more space or a little less space right before the body of the abstract. If you have that problem, you can uncomment the next line, and either add or take away space manually (negative spaces are OK).
%\vspace*{-0.05in}
% The abstract is a summary of the dissertation, thesis, or selected project with emphasis on the findings of the study.  The abstract must not exceed 1 page in length.  It should be printed in the same font and size as the rest of the work.  The abstract precedes the acknowledgments page and the body of the work.


Electrically powered vertical takeoff and landing (eVTOL) aircraft could provide a new mode of air transportation of people and cargo that is low-cost, on-demand, and able to reach more areas than is possible with current technology. They have the unique ability to takeoff and land in congested spaces yet efficiently travel long distances which makes them a promising technology for applications such as rapid medical assistance, automated package delivery, or human transportation. This type of aircraft has only recently become a possibility, owing to advancements in battery technology, computing power, and sensor technologies, and thus support for eVTOLs is lacking among high-fidelity graphics simulation software. High-fidelity graphics are important for the goal of fully autonomous eVTOL aircraft in order to accurately simulate vision-based navigation using camera sensors.

In this work, we present \textit{VTOL-AirSim}, an extension of Microsoft AirSim with full integration of a tiltrotor eVTOL aircraft. Built on Unreal Engine, a high-end graphics game engine, it includes photorealistic graphics for simulated camera images and for high-quality presentations. We created the visual components of a fully animated tiltrotor vehicle and a detailed city environment for it to fly in. The tiltrotor can be controlled via motor PWM commands, by overriding its state in the world with the use of an external dynamics simulation, or by using the PX4 Autopilot. We give a tutorial on how to use VTOL-AirSim where we provide examples for each method of control, including scripts for controlling the tiltrotor via a geometric controller and control allocation module developed by others in the BYU MAGICC Lab. We also show how to further develop VTOL-AirSim, and how to add custom aircraft or custom environments so that others may use it in their own research.