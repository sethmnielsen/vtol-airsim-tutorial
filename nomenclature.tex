%\chapter*{Nomenclature}
%
%\thispagestyle{plain}
%\pagenumbering{roman}
%\setcounter{page}{15}
%
%\addcontentsline{toc}{chapter}{Nomenclature}
%\label{ch:Nomenclature}
%
%

%\begin{nomenclature}
\begin{tabbing}
12345678 \= \kill

$B$             \> Barrier to extract information about a product from the product itself\\
$\overline{D}$  \> Macroscopic strain rate\\
$D_{0}$                 \> First component of strain rate tensor\\
$D^{N}_{k}$         \> Normal direction of the $k$-th lamina, also an axis for the lamina reference frame\\
$D^{R}_{k}$         \> Rolling direction of the $k$-th lamina, also an axis for the lamina reference frame\\
$D^{T}_{k}$         \> Transverse direction of the $k$-th lamina, also an axis for the lamina reference frame\\
$\overline{d}$  \> Average grain size\\
$\Delta g$          \> Volume of discretized bins in Fundamental Zone\\
$E$                         \> Young's modulus\\
$E_m(wxyz)$     \> Fourier coefficients representing Young's modulus in the $wxyz$ direction for the\\
                                \> $m$-th bin of the Fundamental Zone\\
$F$             \> Estimated rate at which information is extracted from a product\\
$F_m$               \> Fourier coefficients of crystal volume fraction in the\\
                \> $m$-th bin of the Fundamental Zone\\
$G$                         \> Shear modulus\\
$g$                     \> Euler angles from Sample to Crystal reference frames\\
$g_{wx}$            \> Orientation matrix of Euler angles from Sample to Crystal reference frames\\
$\dot \gamma$       \> Shear rate\\
$\dot \gamma _0$ \> Reference shear rate\\
$K$             \> Estimated or actual information contained by a product\\
$L$                     \> Distance between straight, parallel lines used to determine average grain size\\
$\lambda$       \> The contraction ratio for the strain tensor\\
$M$                     \> Material class, (e.g., nickel, copper)\\
$M_0$                       \> Selected alloy from material class\\
$N$                         \> Number of laminae to be used in layer-by-layer creation of material\\
$n$                         \> Inverse rate sensitivity parameter\\
$n_c$               \> Number of columns in the binned Fundamental Zone\\
$n_h$               \> Number of layers in the binned Fundamental Zone\\
$n_r$               \> Number of rows in the binned Fundamental Zone\\
$\nu$                       \> Poisson's ratio\\
$P$             \> Estimated power exerted to extract information contained by a product\\
$\phi_{1,i}$                  \> Lamination orientation for the $i$-th layer\\
$S$             \> A measure of a product's ability to contain information\\
$S_{11}$            \> Material property constant obtained from literature for selected material class\\
$S_{12}$            \> Material property constant obtained from literature for selected material class\\
$S_{44}$            \> Material property constant obtained from literature for selected material class\\
$\overline{S}(wxyz)$     \> Sample compliance (average crystal compliance)\\
$s$                         \> Slip systems. Comprised of slip plane normals, $\{111\}$, and slip directions $<110>$\\
$\sigma '_{ij}$ \> Deviatoric stress\\
$\sigma_y$          \> Yield strength\\
$T$             \> Estimated time to extract information $K$\\
$t$             \> Reference time frame for reverse engineering a product\\
$\tau$          \> Reference time frame when all parameters are known\\
$\tau_0$            \> Lattice friction stress\\
$\tau^*$                \> Reference shear stress\\
$Y_m$               \> Fourier coefficients representing yield strength physics\\
%\end{tabbing}
%\begin{tabbing}
%12345678 \= \kill
\textbf{Subscripts, superscripts, and other indicators}\\
$[\hspace{.05in}]^*$          \> indicates total measure or effective property\\
$[\hspace{.05in}](t)$         \> indicates $[\hspace{.05in}]$ is a function of time, in the $t$ domain\\
$[\hspace{.05in}](\tau)$      \> indicates $[\hspace{.05in}]$ is a function of time, in the $\tau$ domain\\
$[\hspace{.05in}]_0$          \> indicates $[\hspace{.05in}]$ is evaluated at time $t$ or $\tau$ equal to zero\\
$[\hspace{0.05in}]_p$         \> indicates $[\hspace{.05in}]$ is in the part reference frame\\
$[\hspace{0.05in}]_c$         \> indicates $[\hspace{.05in}]$ is in the crystal reference frame\\
$[\hspace{0.05in}]_l$         \> indicates $[\hspace{.05in}]$ is in the lamina reference frame\\
$[\hspace{0.05in}]_t$         \> indicates $[\hspace{.05in}]$ is the target value
\end{tabbing} 